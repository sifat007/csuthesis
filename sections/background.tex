\chapter{Background}
\label{chap:background}
We now describe the background needed to make the paper self contained.  We
also present prior related work on the problem.

%\subsection{Dynamic Programming Solution of 0/1-KPDP}
% Taken From: 0/1 Knapsack on Hardware: A Complete Solution (Background section)

The 0/1 Knapsack Problem (KP) is formally
defined~\cite{Kellerer-etal-KP-book-2004} as follows.  Given a set of $N$
items, each  a known weight and a knapsack with a limited capacity $C$,
where each item $i$ consists of the profit $p_i$ and weight $w_i$ , the
objective of the 0/1-KP is to select items to achieve the maximum total profit
without exceeding the capacity.  Mathematically,

\begin{equation}
  \begin{array}{ll}
 \text{Maximize} & \quad {\displaystyle\sum^N_{i=1} p_i x_i} \\[5mm]
\text{subject to}& \quad {\displaystyle\sum^N_{i=1} w_i x_i < C} \\
 & \quad x_i \in \{0,1\}
  \end{array}
\end{equation}

% The general dynamic programming approach to the knapsack problem is to
% construct a table that is as wide as the capacity of the knapsack, and as
% high as the cardinality of the set of items to choose from. Starting from
% the element of the table that corresponds to the first item and a capacity
% of 1, the table is completed by computing elements of the table according to
% the recurrence equation:

%\subsection{Existing algorithms for solving 0/1-KP}

There is a significant body of work on algorithms and tools for KP, many of
them described in standard textbooks~\cite{MT90, Hu-IPNF-book-1969,
  Kellerer-etal-KP-book-2004, Cormen-etal-Algorithms-2009}.  Being an NP hard
problem, many authors investigate heuristic and/or approximate algorithms,
with or without bounds on the approximations.  As for exact solutions, there
are two main algorithms: dynamic programming (DP) and branch-and-bound (BB).
We focus on DP, which may be defined via the following recurrence.
% First approach is to find the exact optimal solution which in worst case
% will always have exponential time complexity. And the second approach is to
% find approximate solutions that have significantly less time-complexity but
% will not find the optimal solution every time.  Depending on the real-world
% problem requirements either type of solution can be used.

% Among the exact optimal solutions Dynamic Programming(DP) and
% Branch-and-Bound(BB) solutions are widely known.

\begin{equation} \label{eq:KPDP}
  \begin{array}{l}
    f(i,j) =  \\
    \left\{
    \begin{array}{ll}
      0,  & \text{if } i=0 \text{ or }k=0\\
      f(i,j-1),        & \text{if } i>0 \text{ and } j<w_i\\
      \max(f(j,i-1),\\
      p_i + f(j-w_i , i-1))   & \text{if } i>0 \text{ and } j \geq w_i
    \end{array} \right.
  \end{array}
\end{equation}

The function $f(i,j)$ denotes and defines the maximal profit that can be
achieved with a knapsack of capacity $j$, drawing items out of only the first
$i$ items.  The standard KPDP algorithm computes and stores this table in an
order (possibly in parallel) ditated by the dependences of the recurrence
doing $O(NC)$ work, and then performs an $O(N)$ work ``backtracking''
traversal of the table in order to construct the actual solution.  The total
execution time of KPDP is thus $O(NC)$

In this paper, we exploit two known improvements to KPDP.  The first one
reduces the space complexity to only $O(C)$ so that the whole table does not
need not be saved (this comes with a two-fold increase in execution time). For
the sake of completeness, we describe them here.

\section{Memory efficient KPDP}

The dynamic programming algorithm normally requires the entire table to be
retained, and this is often unaccpetably large for large problems.  The memory
efficient KPDP is an adaptation of an old trick to improve DP algorithms, that
was first proposed by Hirshberg for the longest common subsequence (LCS)
problem~\cite{Hirschberg-CACM-1975}.  The technique, described by Kellerer et
al.~\cite[pp 46-50]{Kellerer-etal-KP-book-2004} and by Nibbelink et
at.~\cite{sanjay-asap07} who also propose a hardware implementation on FPGAs,
avoids backtracking by recursively, \emph{directly solving} subproblems of the
form $\langle i, j, c \rangle$, that determine the optimal subset of the
$i$-through-$j$-th (inclusive) objects for a capacity of $c$.  The top level
call is thus to $\mathbf{Solve} \langle 1, N, C \rangle$, and
$\mathbf{Solve} \langle i, j, c \rangle$ proceeds as follows:

\begin{enumerate}
\item \textbf{Base case:} if $i=j$, there is only one object, so choose it if
  and only if it fits (i.e., $w_i\leq c$)
\item \textbf{General case:} Let $m=\lfloor\frac{j-i}{2}\rfloor$.  Determine
  (details below) a $c^*$ between 0 and $c$ such that in the (some) solution
  to the subproblem $\langle i,j,c\rangle$ the combined sum of the weights of
  the selected object in the ``first half'' i.e., $i$-through-$m$-th objects,
  is $c^*$.  This is called the ``optimal split'' of the capacity $c$.  Now,
  \begin{enumerate}
  \item Recursively call $\mathbf{Solve} \langle i, m, c^* \rangle$, and
  \item Recursively call $\mathbf{Solve} \langle m+1, j, (c-c^*) \rangle$
  \end{enumerate}
\end{enumerate}

So the main issue is to determine $c^*$.  We first determine for the first
``half'' of the objects, a vector $X[j]$ for $0\leq j\leq c$, the optimal
profit that can be obtained with capacity $j$ (using only objects in that
half).  We also construct a similar vector, $X'[j]$ for the other half.  This
can be done in only $2c$ memory using recurrence~(\ref{eq:KPDP}), by maintianing
only two rows of the table, and costs $(j-i)c$ work.  Using these two arrays,
$c^*$ is given by
\[ c^* = \mathrm{argmax}_{i=0}^{c} X[i]+X[c-i]
\]

The capacity arguments in all the calls at any levels of the recursion tree
add up to $c$, and the number of objects considered halves level by level.
Hence, the total work of this algorithm is $2NC$, just a two-fold increase.

% \begin{flushleft}
% \noindent\rule{9cm}{1pt}
% \textbf{Algorithm 1} Classic Backtracking
% \noindent\rule{9cm}{1pt}
% \end{flushleft}
% \begin{algorithmic}
% \STATE $cap = C; i = N$
% \WHILE {$i > 0$}
% 		\STATE $bitV ector = booleanT able[cap]$
% 		\STATE MASK OUT rightmost [$(i + 1)$to $N$ ] bits of $bitVector$
% 		\STATE $i$ = index of rightmost 1 bit in ( MASKED ) $bitVector$
% 		\STATE add $i$ to solution list
% 		\STATE $i = i - 1$
% 		\STATE $cap = cap - Weight[i]$
% \ENDWHILE
% \end{algorithmic}
% \noindent\rule{9cm}{1pt}

\section{Sparse KPDP (SKPDP)}

Andonov and Rajopadhye~\cite{sanjay-asap96} and Dupont de Dinechin et
al.~\cite{sanjay-irreg96} showed how, for the 0/1-KPDP and the UKP (the
unboundend knapsack problem), the sequences representing the $i$-th row could
be computed for the one representing the $(i-1)$-th row.  The work is similar
to early ideas on a list based algorithm~\cite{Horowitz-Sahni-JACM-1974} that
reduced the complexity of the subset-sum problem to $O(\sqrt{2^N})$, but uses
\emph{streams} of \emph{index-value pairs}.  The main idea behind the SKPDP is
the notion of monotomicity: it is clear that for any fixed $i$, the values of
$f(i,j)$ are monotomically non-decreasing with respect to $j$.  This is
because, in any subptooblem, we can do no worse than we are currently doing by
increasing capacity.  This is illustrated in Table~\ref{tab:sparsity1} which
shows the DP table for a KP instance where the four items have weights,
$5, 4, 6, 1$ and profits, $7, 8, 9, 4$, respectively.  The boxed entries are
the first occurrence of a value in a given row.  Because of sparsity, one may
enviage a ``sparse'' representation, where all the table entries are not
computed, rather only the ``points of inflection'' are explicitly evaluated.
To do this, we use a representation for the table in the form of
``index-value'' pairs: the entire $i$-th row is represented by a list, of the
form, $\langle j, f(i,j)\rangle$, as shown in Table \ref{tab:sparsity2}.

\begin{table}[htbp]
\caption{A 0/1-KPDP Table}
\begin{center}
\begin{tabular}{*{12}{|c}|}
  \hline
  \textbf{Items}&\multicolumn{10}{|c|}{Capacity} \\
  \cline{2-12} & \B{0} & \B{1}& \B{2}& \B{3}& \B{4}& \B{5}& \B{6}& \B{7}& \B{8}& \B{9}& \B{10} \\
  \hline
  \textbf{1}& 0 & 0& 0& 0& 0& \boxed{7}& 7& 7& 7& 7& 7 \\
  \hline
  \textbf{2}& 0 & 0& 0& 0& \boxed{8}& 8& 8& 8& 8& \boxed{15}& 15 \\
  \hline
  \textbf{3}& 0 & 0& 0& 0& \boxed{8}& 8& \boxed{9}& 9& 9& \boxed{15}& \boxed{17} \\
  \hline
  \textbf{4}& 0 & \boxed{4}& 4& 4& \boxed{8}& \boxed{12}& 12& \boxed{13}& 13& \boxed{15}& \boxed{19} \\
  \hline
\end{tabular}
\label{tab:sparsity1}
\end{center}
\end{table}

\begin{table}[htbp]
\caption{A SKPDP Table}
\begin{center}
\begin{tabular}{|c|l|}
\hline
\textbf{Items}& \\
\hline 
  \textbf{1}& $\langle 0, 0\rangle$, $\langle 5, 7\rangle$  \\
  \hline
  \textbf{2}& $\langle 0, 0\rangle$, $\langle 4, 8\rangle$, $\langle 9, 15\rangle$ \\
  \hline
  \textbf{3}& $\langle 0, 0\rangle$, $\langle 4, 8\rangle$, $\langle 6, 9\rangle$, $\langle 9,
              15\rangle$, $\langle 10, 17\rangle$ \\
  \hline
  \textbf{4}& $\langle 0, 0\rangle$, $\langle 1, 4\rangle$, $\langle 4, 8\rangle$, $\langle 5,
              12\rangle$, $\langle 7, 13\rangle$, $\langle 9, 15\rangle$,
              $\langle 10, 19\rangle$ \\
  \hline
\end{tabular}
\label{tab:sparsity2}
\end{center}
\end{table}

\section{Add-Merge-Kill}

Andonov and Rajopadhye~\cite{sanjay-asap96} and Dupont de Dinechin et
al.~\cite{sanjay-irreg96} proposed to implementat the SKPDP algorithm on an
early form of hardware accelerator called a WAP (Wavefront Array
Processor~\cite{kung-sy-vlsi, kung-sy-etal}).  However, because the memory
efficient technique had not been discovered at that time, it remained an
unrealisitic paper design, because the total number of data transfers that the
accelerator needed to perform was the same order as the number of
computations.

The technique Add-Merge-Kill (AMK) is the core of SKPDP that generates the sparse table row-by-row. Let's take the 3\textsuperscript{rd} row of the Table (\ref{tab:sparsity2}),
\begin{equation}
\label{eq:list1}
\langle 0, 0\rangle, \langle 4, 8\rangle, \langle 6, 9\rangle, \langle 9, 15\rangle, \langle 10, 17\rangle
\end{equation}

This list of pairs (sparse row) is representative of the 3\textsuperscript{rd} row of the Table (\ref{tab:sparsity1}), that represents the maximum profits of taking the first 3 elements of the problem instance given capacities ranging from $1$ to $10$. From the 3\textsuperscript{rd} row to generate the 4\textsuperscript{th} row that also takes the 4\textsuperscript{th} element of the problem instance into consideration we first add the the weight and profit of the 4\textsuperscript{th} element, i.e. $\langle 1,4 \rangle$, to each pairs of the list (\ref{eq:2}).
We get,
\begin{equation}
\label{eq:list2}
\langle 1, 4\rangle, \langle 5, 12\rangle, \langle 7, 13\rangle, \langle 10, 19\rangle, \langle 11, 21\rangle
\end{equation}

Now applying the algorithm \ref{algo:MK} below to the lists (\ref{eq:list1}) and (\ref{eq:list2}) to get the 4\textsuperscript{th} row. 

\begin{equation}
\label{eq:list3}
\langle 0, 0\rangle, \langle 1, 4\rangle, \langle 4, 8\rangle, \langle 5,
              12\rangle, \langle 7, 13\rangle, \langle 9, 15\rangle,
              \langle 10, 19\rangle
\end{equation}

\begin{algorithm}

\SetAlgoLined
\KwResult{(i+1)-th row: $S_{i+1}$}
 $S_{i} \leftarrow $ current row\; 
 $S_{i+1} \leftarrow $ empty\;
 $S'_{i} \leftarrow $ $\langle weights[i], profits[i] \rangle$ added to each pairs of $S_i$\;
 $j,k,p \leftarrow 0,0,0$\;
 \While{the end of $S_i$ not reached}{
  \uIf{$S_i[j].weight < S_i[k].weight$}{
   	\eIf{$S_i[j].profit < S_i[k].profit$}{
   		$S_{i+1}[p] \leftarrow  S_i[j]$\;
   		$p \leftarrow p+1$\;
   		$j \leftarrow j+1$\;
   	} 	
	{
   		$k \leftarrow k+1$\;   	
   	}
   }
   \uElseIf{$S_i[j].weight > S_i[k].weight$}{
   	\eIf{$S_i[j].profit > S_i[k].profit$}{
   		$S_{i+1}[p] \leftarrow  S_i[k]$\;
   		$p \leftarrow p+1$\;
   		$k \leftarrow k+1$\;
   	} 	
	{
   		$j \leftarrow j+1$\;   	
   	}
  }
  \Else{
   	\eIf{$S_i[j].profit > S_i[k].profit$}{
   		$k \leftarrow k+1$\;
   	} 	
	{
   		$j \leftarrow j+1$\;   	
   	}  	
  }
 }
 \caption{Merge-Kill \label{algo:MK}}
\end{algorithm}




