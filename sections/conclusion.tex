\chapter{Conclusion}

\label{chap:conclusion}
According to our experiments many knapsack problem instances can benefit from utilizing the inherent sparsity of a KPDP. The fact that for many realistic problem instances SKPDP is exponentially better than conventional KPDP, is a significant discovery. We also show the different attributes of a knapsack problem instance that effects the sparsity of SKPDP. We propose two parallelization techniques and present initial findings of those parallel implementations. Other than the two proposed parallelization techniques we also explored another technique called ``Rank Convergence'' and proved that KPDP can not be parallelized using that technique. The next big step towards making SKPDP more useful is to find an analytic model to predict whether using SKPDP on a given knapsack problem instance is advantageous or not. Also, more effort is needed to improve the palatalization of SKPDP.





